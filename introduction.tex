\section{Introduction}

% ¶ para 1 - new application, like AR / RAN (WiFi does work)
% new solution: camera-display comm. or NFC
%to be modified...
Visible light communications (VLC) have gained wide attention in the research community recently.
The possibilities of using every light emitting diode (LED) that already exists in our daily life to talk and to serve as
a new pervasive communication infrastructure are endless - it enables interference-free, secure, low cost, and energy efficient short range communications to a wide range of electronic devices and appliances, from cars with LED taillights, light fixtures with LED light bulbs, to refrigerators with tiny LED indicators. VLC uses the optical signal to carry digital information by controlling the LED's light intensity in free space. When the light is transmitting, due to persistence of vision, human eyes cannot perceive the transmissions. 

% ¶ para 2 - VLC is a new communication technique fitting these visible requirement (M: Maybe CamCom is a better term?)
% Existing works: either only send repetitive signal or only allow indirect communication, or too slow for 30 fps camera (< 15 bit/s)
%to be modified...
This work implements a form of VLC that uses a \textbf{single LED} as the transmitter and a Complementary Mental-Oxide-Semiconductor (CMOS) \textbf{rolling shutter camera sensor} as the receiver, a form of Camera Communications (CamCom).
We chose to develop this form of VLC due to the universal availability of both the transmitting component - a simple LED (as opposed to an array of LEDs or an LCD screen), and the receiving component - a CMOS camera. 
Almost every modern mobile device, including smartphones, tablets, and laptops, is equipped with at least one built-in CMOS camera. 
We believe that this would be the best paradigm to jump-start the VLC technology in the market.
As a result, one of the most important objectives of our work is to ensure the compatibility of a wide range of unmodified CMOS cameras to the LEDs transmission.  


% ¶ para 3 - LOS VLC/CamCom (compared to NLOS VLC/CamCom) is better because 
% (1) gives the system the capability to perform visual association (of data and identity, AR) 
% (2) enable data rate aggregation with multiple lights or gives better spatial reuse efficiency (multiple access), utilizing the high spatial resolution of the camera sensor 
% (3) much much better SNR -> longer range
%to be modified...
One very important advantage of CamCom is that the acquired images during the receiving process provide a means to \textbf{visually associate} the transmitting object with the transmissions, since the receiver has the knowledge of which pixels in the images are transmitting. \autoref{fig:scenario} shows a simple example, the user has the idea to know that every light is talking independently. We can use the idea in many applications. One is \textbf{Advanced Driver Assistance System (ADAS)}. we can use the LED tail lights of the front car to transmit a message to the camera of the rear cars. The message can contain safety-related status information, such as speed, location, heading, and brake/ turn signal status, or user-specified short text messages to form an advanced driver assistance system. 
Another example is \textbf{indoor positioning}. We receive the light ID transmitted by the LED light and obtain the vertical distance between the receiver and the transmitting lights, and the horizontal distances between the lights. Then, we can calculate the relative position and know where we are in the building. 
We also can use the LED lights for \textbf{advertising}. For example, in a supermarket, we can use the LED light on the shelf to transmit the sale and discount information of the goods. We can use the traffic lights and signs to tell us the road status ahead. 
Besides, we can apply this kind of communication for \textbf{system diagnosis}. For example, when a scooter breaks down, we can transmit the sensor log via the LED tail lights to briefly check and detect the problem of the scooter; in a server room, full of machines on the racks, it is often hard to find out which server is having trouble, the nature of the problem, and the IP address of the machine. Imagine if each server has VLC capability on one of its high brightness LED, then we can point our smartphone's camera at an array of servers, and immediately these information will be shown right next to the server, for the administrator to diagnose the problem in an easier way.  

\begin{figure}[!t]
	\includegraphics[scale=0.075]{pic/scenario.JPG}
	\centering
	\caption{An example of utilizing CamCom for visual association.}
	\label{fig:scenario}
\end{figure}

% Challenge 1:
% sync (repetitive signal does need to deal with it because data is encoded in only a single image frame)
% NLOS: easy to fix sync, low signal sample loss prob., LOS: difficult: patterns are incontinuous
% so we have sync mechanism
Challenge 1:
sync (repetitive signal does need to deal with it because data is encoded in only a single image frame)
NLOS: easy to fix sync, low signal sample loss prob., LOS: difficult: patterns are incontinuous
so we have sync mechanism 
%to be modified...
As the transmitting light and the receiving camera are not synchronized, the transmitting frame rate and the receiving frame rate are not usually the same.
When the transmitting frame rate is higher than the receiving frame rate, there could be symbol loss.
When the transmitting frame rate is lower than the receiving frame rate, reception of redundant symbols is possible. 
Although no information is lost, the receiver still needs to detect the redundant symbol so that it can be dropped to obtained the correct symbol sequence.
A more common event when the transmitting frame duration and the receiving frame duration are different is that there could be multiple image areas each corresponding to a symbol. In other word, a mix of different symbols in a single received image frame. 
A mixed frame is a very common and periodic event.
Even when the transmitting and the receiving frame durations are very close to each other, mixed frames would be received because of the phase difference between transmitter and receiver.Therefore, in this paper we also present a scheme that can handle \textbf{unsynchronized} transmitter and receiver pairs. 

% Challenge 2:  compatibility + high data rate
% one-way communication, but we want a higher rate → rolling shutter sampling (FSK) → 1/t_r is sampling rate, but different for different cameras → Read-out-time estimation for compatibility among different cameras
% high data rate, two possibility (1) high symbol rate → this is not possible, as a large portion of signal samples can be lost (2) high dimension modulation (high number of kinds of symbols) → we need accurate and fast frequency estimation algorithm (YIN and variants, compared to FFT and general DIP results)
% (??) impact of exposure time: unfixable: we do measurement?
Challenge 2:  compatibility + high data rate
one-way communication, but we want a higher rate → rolling shutter sampling (FSK) → 1/tr is sampling rate, but different for different cameras → Read-out-time estimation for compatibility among different cameras
high data rate, two possibility (1) high symbol rate → this is not possible, as a large portion of signal samples can be lost (2) high dimension modulation (high number of kinds of symbols) → we need accurate and fast frequency estimation algorithm (YIN and variants, compared to FFT and general DIP results)
(??) impact of exposure time: unfixable: we do measurement? 
%to be modified...
In this work, we presented a \textbf{theoretical model} to describe the single-LED-to-rolling-shutter-camera channel. Furthermore, we proposed the \textbf{Rolling Shutter-Frequency Shift Keying (RS-FSK) modulation} and demonstrated that the modulation is compatible to a wide range of CMOS cameras with different resolution, read-out time, and exposure time, satisfies the lighting requirements, and can be implemented with cost-efficient hardware. 

% ¶ para last - paper arrangements (optional) 
%to be modified... or not going to use
This paper proceeds as follows.
Chapter 2 presents some prior research done in this area.
Chapter 3 provides the background of the channel model of the rolling shutter camera, and RS-FSK modulation and demodulation techniques.
Chapter 4 states the problem caused by unsynchronized transmitter and receiver pairs.
Chapter 5 presents the hardware component and the software design of our system.
Chapter 6 presents our experiment and evaluation results. 
Finally, chapter 7 gives the concluding remarks of this thesis. 


% \begin{figure}[!htb]
% 	\includegraphics[scale=0.15]{fig/virtual_switch_hor.png}
% 	\centering
% 	\caption{An example of utilizing CamCom for visual association.}
% 	\label{fig:virtual_switch}
% \end{figure}

% Figure~\ref{fig:virtual_switch} shows a simple example. A user points her smartphone camera at multiple lights, touches the screen to select one, and uses the virtual switch that appears on the screen to adjust its lighting level. 
% In this scenario, the lights simply periodically transmit simple identifiers, such as IP addresses, to the smartphone. 
% The system is able to directly associate what the user sees (the image pixels mapped to the light selected by
% the user) to the transmission (the identifier of the light), and uses WiFi to sends the actual command to the right light to adjust the setting. 
% Compared to the conventional approach of identifying a particular set of visual features unique to the object with computer vision techniques, this approach is both simpler and more accurate.

% \begin{figure}[!htb]
% 	\includegraphics[scale=0.7]{fig/v2v.png}
% 	\centering
% 	\caption{An example of utilizing VLC for Vehicle communication.}
% 	\label{fig:v2v}
% \end{figure}