\begin{abstract}
%縮字數
The paper presents RollingLight, a visible light communication (VLC) system that utilizes a single LED light source as the transmitter and a CMOS rolling shutter camera as the receiver. 
In our system, we propose the use of Rolling Shutter Frequency Shift Keying (RS-FSK), a very simple modulation that can be implemented with very low cost embedded microprocessors, thus enabling LEDs that already exist in our daily life, such as LED indicators in electronics and light fixtures, to ``talk''. 
Today, almost every modern mobile device is equipped with at least one built-in CMOS rolling shutter camera, and it can receive signals from the LEDs without any modification, and thus, any additional cost. 
Our idea is to use the acquired images from the rolling shutter camera to extract the message transmitted from the LED. This technology can be used in many applications, such as advanced driver assistance system (ADAS), indoor positioning, advertising, and problem diagnosis. Our system allows rolling shutter cameras with different hardware specifications such as resolution, frame rate, read-out time, and exposure time, to all have the capability to decode the transmission, and the decoding performance scales with the camera specifications. In addition, the transmission is not visible to human eyes, thus keeping the original illumination or signaling function of the LEDs. Our design is evaluated with experiments performed with a range of different cameras. The RS-FSK modulation scheme can be used to transmit data at 11.25 bytes per second. We also develop a real-time application on the smartphone.

% VLC is a new data transmission technology. Unlike traditional RF wireless communication, VLC uses the optical signal to carry digital information by controlling the LED's light intensity in free space. VLC has the properties of high directivity, high security and high bandwidth.
% In recent years, with the development of the LED technology, due to its advantages of low power consumption, high efficiency, and long life, LED has gradually replaced the traditional fluorescent lighting. The quick response time and the low cost also make LED a very attractive transmitter solution for communications. 

% In our proposed system, we use CMOS rolling shutter camera as the receiver. Today, almost every modern mobile device is equipped with at least one built-in CMOS camera, and CMOS camera can receive signals from the LEDs without any modification, and thus, any additional cost. 
% Our idea is to use the acquired images to extract the message transmitted from the LED. This technology can be used in many applications, such as advanced driver assistance system (ADAS), indoor positioning, and visual-associated application.

% Due to the wide range of frames rates of the cameras, it is common to receive redundant symbols or have symbol loss; Frames with more than one data symbols, i.e., mixed frames, may also happen because of the phase offset between the transmitter and the receiver.
%  Both of the issues are due to the unsynchronized nature of our proposed system. In this paper, we propose a number of schemes to address the issues caused by unsynchronized transmitter and receiver.
% To evaluate the feasibility of these schemes, we use software-defined radio (SDR) to implement the transmitter and use the built-in camera of several recent smartphones and an industrial camera as the receiver. We also develop a real-time decoding application on the smartphone. Our result shows that we can improve the packet reception rate (PRR) from 0.6 to 0.99 when the transmitting and receiving frame rates are close. We also can improve the PRR from 0.0 to 0.8~1.0 even when the transmitting and receiving frame rates have a large difference. The overall throughput can reach 18 bytes per second. We expect this technology can be widely used for a wide range of applications in the future.
\end{abstract}

